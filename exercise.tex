\documentclass[a4paper,11pt,dvipdfmx]{ujarticle}
% パッケージ
\usepackage{graphicx}
\usepackage{url}
% レイアウト指定を記述したファイルの読み込み
\input{layout}

% タイトルと氏名を変更せよ.
\title{日本におけるデジタル化の状況}
\author{長谷川 蒼太}

\begin{document}

\maketitle %ここにタイトルが入る

% ここから本文
\section{ブロードバンドの整備状況}
OECDによるブロードバンド回線の普及に関する調査\cite{oecd}によると、図\ref{fig:myFig}に示すように、日本における 100人あたりのモバイルブロードバンドの加入者数は190.5で、第1位になっている。2位はエストニア
で、3位米国と続く。

% 本文(1)

%  図番号の参照: \ref{}
% を使う
% 文献データベースのキーワードは oecd と imd
% になっている.

% 図の挿入
\begin{figure}[htbp]
    \centering
    \includegraphics{/Users/hasegawasota/Downloads/fig21.png}
    \caption{光ファイバー回線の加入者数(100人あたり)}\label{fig:myFig}
\end{figure}
% を
% \begin{figure}[htbp]
% \end{figure}
% で囲み
% \caption{}
% で図のタイトルを入れる.
% \label{}
% を使って図番号が参照できるようにする
% また,
% \centering
% で図が中央に来るようにする

% ーーー
% 節見出し(2)

% 本文(2)
\section{デジタル競争力ランキング}
国際経営開発研究所(IMD)の調査\cite{imd}によると、表\ref{tbl:myTbl}にすように、日本のデジタル競争力のランキ
ングは調査対象の64カ国中、総合で28位、知識分野で25位となっている。
% 表の挿入
\begin{table}[htbp]    
    \centering
    \caption{デジタル競争力ランキング(64カ国中)}\label{tbl:myTbl}
    \begin{tabular}{|c|c|c|}
        \hline
        国  & 総合 & 知識 \\
        \hline
        米国  & 1位 &3位 \\
        \hline
        香港  & 2位 & 5位 \\
        \hline
        スウェーデン  & 3位 & 2位 \\
        \hline
        デンマーク  & 4位 & 8位 \\
        \hline
        シンガポール  & 5位 & 4位 \\
        \hline
        \hline
        韓国  & 12位 & 15位 \\
        \hline
        中国  & 15位 & 6位 \\
        \hline
        \hline
        日本 & 28位 & 25位 \\
        \hline
    \end{tabular}
\end{table}      
% \end{tabular}    
% による表の記述を 
% \begin{table}[htbp]
% \end{table}
% で囲み
% \caption{}
% で表のタイトルを入れる.
% \label{}
% を使って表番号が参照できるようにする
% また,
% \centering
% で表が中央に来るようにする

% ーーー
% 見出し(3)
\section{考察}
\begin{itemize}
    \item 日本について
    \begin{itemize}
        \item モバイルブロードバンドの加入者数は100人あたり約190.5人と、調査対象国の中で最も高い水準にあり、インフラの整備が非常に進んでいる
        \item デジタル競争力ランキングでは、64か国中28位で特に「知識分野」においては25位と、インフラの整備状況に比べて総合的な競争力が低くICT人材の不足や、教育・企業におけるデジタル活用の遅れが一因と考えられる
    \end{itemize}
    \item 全体
    \begin{itemize}
        \item モバイルブロードバンドの加入者数とデジタル競争ランキングに比例の関係はない
    \end{itemize}
\end{itemize}
% を使って箇条書きで記述する

% ここに参考文献が入る

\cite{oecd}
\cite{imd}
\bibliographystyle{junsrt}
\bibliography{exercise.bib}

\end{document}